\section{Physik der Gitarre}

\subsection{Vorwort}


\begin{figure}[h]
	\centering
	\includegraphics{bilder/gitarre_aufbau.jpg}
	\caption{Aufbau einer Konzertgitarre}
\end{figure}

Am Griffbrett der Gitarre mit den Bundstäbchen kann man Töne abgreifen. Der Abstand zwischen zwei Bundstäbchen ist dabei ein Halbtonschritt. Dadurch können identische Töne auf verschiedenen Saiten gespielt werden.

\begin{table}[H]
\centering
\begin{tabular}{c|cccccc}
Saite & 1 & 2 & 3 & 4 & 5 & 6 \\
\hline
Note & e' & b & g & d & A & E \\
\hline
Frequenz / Hz & 329.63 & 246.94 & 196.00 & 146.83 & 110.00 & 82.41  
\end{tabular}
\caption{Frequenzen der Leersaiten}
\end{table}

Durch die Überlagerung zweier Schwingungen ähnlicher Frequenz kommt es zu Schwebungseffekten. Dabei ändert sich die Amplitude der resultierenden Schwingung periodisch. Für zwei Schwingungen gleicher Amplitude gilt folgendes Additionstheorem
$$\sin(\omega_1t)+\sin(\omega_2t) = 2\sin\left( \frac{\omega_2+\omega_1}2t \right) \cos\left( \frac{\omega_2-\omega_1}2t \right)$$
Es ergibt sich eine Schwebungsfrequenz $\omega_S$ und eine resultierende Frequenz $\omega_R$ gemäß
$$\omega_S = \frac{\omega_2-\omega_1}2 \hspace{0.5cm} \text{ und } \hspace{0.5cm} \omega_R = \frac{\omega_2+\omega_1}2.$$
Dieses Phänomen wird im ersten Teilexperiment untersucht.
Beim Anschlagen einer Gitarrensaite breiten sich Transversalwellen in beide Richtungen aus und werden an Steg und Sattel reflektiert. Dadurch bilden sich stehende Wellen auf der Saite aus. 

\begin{figure}[h]
	\centering
	\includegraphics{bilder/StehendeWelle.jpg}
	\caption{Stehende Welle und deren Harmonische bis $n=5$}
\end{figure}

Da die Stehenden Wellen an den Enden der Saite jeweils Knoten ausbilden, gilt für die Wellenlängen der Moden
$$\lambda_n = \frac{2L}n$$
mit $n\in \mathbb N$ und $L$ als Saitenlänge. Die Auslenkung der Saite ergibt sich dabei als Summe der Moden
$$y(x,t) = \sum_{n=1}^\infty A_n \cos(\omega_nt + \phi_n)\sin(k_nx)$$
mit $k_n = \frac{2\pi}{\lambda_n}$. Der Anschlagpunkt der Saite ist entscheidend für das Auftreten der Moden. Schlägt man die Saite im Abstand $d=\frac Ln$ an, so fehlt die $n$-te Harmonische und ihre Vielfachen. Dies wird im zweiten Teilexperiment untersucht.


\subsection{Versuchsaufbau}

\begin{figure}[h]
	\centering
	\includegraphics[width=\linewidth]{bilder/gitarre_beschriftet.jpg}
	\caption{Versuchsaufbau}
	\label{aufbauGitarreExp}
\end{figure}

Eine Akustikgitarre liegt auf einer Polsterung auf dem Tisch. Das Mirkofon wird mit einem Stativ mittig oberhalb des Schallochs der Gitarre platziert. Das Mikrofon wird im Amplitudenmodus betrieben und mit dem Sensor-CASSY wird die Ausgangsspannung des Mirkofons gemesssen (vgl. Abb. \ref{aufbauGitarreExp}). Vor der Messung wird die Gitarre mit einem Stimmgerät mit Stimmgenauigkeit von $\pm 1 \, \mathrm{Cent}$ gestimmt.


\subsection{Messung der Schwebung}


\subsubsection{Versuchsdurchführung}

Nach dem Stimmen der Gitarre wird die A-Saite leicht verstimmt. Anschließend wird die offene D-Saite zusammen mit der im 5. Bund gegriffenen A-Saite angeschlagen und eine Messung mit dem Mikrofon gestartet. Es werden für vier verschiedene Verstimmungen Messungen aufgenommen. Dabei wird für die erste Verstimmung eine Messung aufgenommen und für die restlichen drei jeweils zwei Messungen. Für die ersten beiden, sowie die letzte Messung hat das Sensor-CASSY folgende Messparameter:
\begin{enumerate}[-]
\setlength{\itemsep}{-5pt} 
\item Messbereich: $-10$ bis $10\,$V
\item Intervall: $200\,\mu$s
\item Messungen: $16000$
\item Messzeit: $3.2\,$s
\end{enumerate}
Für die anderen Messungen werden folgende Messparameter verwendet:
\begin{enumerate}[-]
\setlength{\itemsep}{-5pt} 
\item Messbereich: $-10$ bis $10\,$V
\item Intervall: $500\,\mu$s
\item Messungen: $16000$
\item Messzeit: $8\,$s
\end{enumerate}


\subsubsection{Versuchsauswertung}

Die Rohdaten der Messung sind exemplarisch in Abbildung \ref{rohSch} zu sehen.

\begin{figure}[h]
	\centering
	\includegraphics[width=\linewidth]{plots/rohdaten_schwebung.pdf}
	\caption{Visualisierung der Rohdaten aus der ersten Messung (\mbox{``schwebung\_1.lab''})}
	\label{rohSch}
\end{figure}

Wir bestimmen nun die Schwebungsfrequenz $f_S$ und die Frequenz der resultierenden Schwingung $f_R$ auf zwei unterschiedliche Arten. Zum einen durch Ablesen von Maxima/Minima aus den Rohdaten und zum anderen durch eine Fast Fourier Transformation (FFT) der Rohdaten. 
In der ersten Variante werden die Maxima/Minima per Augenmaß aus Plots der Rohdaten abgelesen, wie es in Abbildung \ref{MaxAblesen} zu sehen ist. Um eine Unabhängigkeit der daraus erhaltenen Werte zu erreichen wird aus den Dateien \mbox{``schwebung\_1.lab''}, \mbox{``schwebung\_2.1.lab''}, \mbox{``schwebung\_3.2.lab''} und \mbox{``schwebung\_4.2.lab''} abgelesen und die FFT auf die Daten aus den Dateien \mbox{``schwebung\_1.lab''}, \mbox{``schwebung\_2.2.lab''}, \\
\mbox{``schwebung\_3.1.lab''} und \mbox{``schwebung\_4.1.lab''} angewendet.

\begin{figure}[H]
	\centering
	\includegraphics[width=\linewidth]{plots/bspMaxima.pdf}
	\caption{Bestimmung der Maxima aus den Rohdaten}
	\label{MaxAblesen}
\end{figure}
Beim Ablesen für die resultierende Schwingung ist zu beachten, dass sich bei einem Nulldurchgang der Schwebungschwingung die Art das Extremums, welches wir ablesen wollen, ändert. Die so abgelesenen Zeitpunkte sind in Tabelle \ref{tabTimeRes} dokumentiert. Angesichts der Messintervalle für die Zeit und der Erkennbarkeit der Extrema wird für die Schwebungen 1 und 2 die Zeitunsicherheit $\sigma_t = 0.2 \, \mathrm{ms}$, für die Schwebung 3 $\sigma_t = 0.5 \, \mathrm{ms}$ und für die Schwebung 4 $\sigma_t = 0.4 \, \mathrm{ms}$ angesetzt.

\begin{table}[H]
\centering

\begin{adjustbox}{width=\textwidth}
\begin{tabular}{c|c|ccccccccccc}
\multirow{4}{*}{1} 
& $n$ & 1 & 20 & 60 & 80 & 100 & 120 & 140 & 160 & 200 & 236 & 256 \\
\cline{2-13}
& $t$ /s & 0.0308 & 0.1614 & 0.4378 & 0.5754 & 0.7136 & 0.8512 & 0.9892 & 1.1276 & 1.4036 & 1.6518 & 1.7902 \\
\cline{2-13}
& $n$ & 276 & 296 & 336 & 376 & 396 & 416 & 436 & & & & \\
\cline{2-13}
& $t$ / s & 1.9278 & 2.0660 & 2.3418 & 2.6178 & 2.7562 & 2.8936 & 3.0322 & & & & \\
\hline
\multirow{4}{*}{2} 
& $n$ & -2 & 17 & 40 & 60 & 100 & 120 & 140 & 160 & 200 & 220 & 240 \\
\cline{2-13}
& $t$ /s & 0.1548 & 0.2848 & 0.4426 & 0.5802 & 0.8548 & 0.9920 & 1.1292 & 1.2662 & 1.5410 & 1.6782 & 1.8156  \\
\cline{2-13}
& $n$ & 280 & 300 & 320 & 340 & 380 & 400 & 420 & & & & \\
\cline{2-13}
& $t$ / s & 2.0900 & 2.2274 & 2.3648 & 2.5022 & 2.7762 & 2.9136 & 3.0510 & & & &  \\
\hline
\multirow{4}{*}{3}
& $n$ & 6 & 25 & 40 & 60 & 80 & 120 & 140 & 160 & 180 & 200 & 220 \\
\cline{2-13}
& $t$ / s & 0.0070 & 0.1370 & 0.2395 & 0.3765 & 0.5135 & 0.7865 & 0.9235 & 1.0605 & 1.11975 & 1.3345 & 1.4710 \\
\cline{2-13}
& $n$ & 240 & 260 & 280 & 300 & 340 & 360 & 380 & 400 & 420 & 440 & 460 \\
\cline{2-13}
& $t$ /s & 1.6080 & 1.7450 & 1.8820 & 2.0185 & 2.2910 & 2.4285 & 2.5655 & 2.7025 & 2.8395 & 2.9765 & 3.1135 \\
\hline
\multirow{4}{*}{4}
& $n$ & 2 & 20 & 40 & 60 & 80 & 100 & 120 & 180 & 200 & 220 & 240 \\
\cline{2-13}
& $t$ / s & 0.0800 & 0.2060 & 0.3446 & 0.4844 & 0.6234 & 0.7636 & 0.9026 & 1.3214 & 1.4604 & 1.5996 & 1.7388 \\
\cline{2-13}
& $n$ & 280 & 300 & 320 & 340 & 363 & 380 & 400 & 420 & 440 & & \\
\cline{2-13}
& $t$ / s & 2.0188 & 2.1576 & 2.2968 & 2.4366 & 2.5966 & 2.7162 & 2.8552 & 2.9942 & 3.1338 & & 
\end{tabular}
\end{adjustbox}
\caption{Abgelesene Zeitpunkte für die Extrema der resultierenden Schwingung}
\label{tabTimeRes}
\end{table}
Da wir einen linearen Zusammenhang $t(n) = Tn+b$ mit der Periodendauer $T$ für die Zeitpunkte erwarten, führen wir jeweils eine lineare Regression durch. Die Ausgabe der Regressionen ist in Tabelle \ref{tabRegressionRes} abgebildet. 

\begin{table}[H]
\centering
\begin{tabular}{c|c|c|c}
Schwebung & $T$ / ms & $b$ / s & $\chi^2$ / $n_{df}$ \\
\hline
1 & $6.89993 \pm 0.00035$ & $0.02354 \pm 0.00009$ & 1.37 \\
2 & $6.86353 \pm 0.00036$ & $0.16829 \pm 0.00009$ & 0.71 \\
3 & $6.84163 \pm 0.00076$ & $0.0340 \pm 0.0002$ & 0.46 \\
4 & $6.97226 \pm 0.00064$ & $0.0660 \pm 0.0002$ & 0.84
\end{tabular}
\caption{Ergebnisse der Regression für die resultierende Schwingung}
\label{tabRegressionRes}
\end{table}
Das $\chi^2/n_{df}$ von Schwebung 3 deutet auf eine zu grobe Fehlerschätzung hin. Da die Unsicherheit der Zeit in diesem Fall dem Messintervall entspricht, wird an der Unsicherheit festgehalten. Die zugehörigen Residuenplots befinden sich in Abbildung \ref{residuenRes}.


\begin{figure}[H]
	\centering
	\includegraphics[width=\linewidth]{plots/residuenRes.pdf}
	\caption{Residuenplots der Regression für die resultierende Schwingung}
	\label{residuenRes}
\end{figure}

Bei Schwebung 2 könnte man eine Systematik vermuten. Es wurden jedoch die Abstände der Extrema bei den Sprüngen nochmals nachgeprüft. Nun werden die Zeitpunkte der Extrema der Schwebungsschwingung bestimmt. Die Zeiten sind in Tabelle \ref{tabTimeSch} aufgeführt. Für die Unsicherheiten der Zeitbestimmung wird für die Schwebungen 1 und 4 $\sigma_t = 0.005 \, \mathrm s$ und für die Schwebungen 2 und 3 $\sigma_t = 0.01 \, \mathrm s$ geschätzt.
\begin{table}[H]
\centering
\begin{tabular}{c|c|cccccccccc}
& $m$ & 1 & 2 & 3 & 4 & 5 & 6 & 7 & 8 & 9 & 10 \\
\hline
1 & \multirow{4}{*}{$t$ / s} & 0.155 & 0.465 & 0.780 & 1.085 & 1.410 & 1.720 & 2.035 & 2.345 & 2.655 & 2.975 \\
\cline{1-1}\cline{3-12}
2 & & 0.38 & 1.02 & 1.66 & 2.33 & 2.99 & & & & & \\
\cline{1-1}\cline{3-12}
3 & & 1.30 & 2.82 & & & & & & & & \\
\cline{1-1}\cline{3-12}
\multirow{3}{*}{4}
& & 0.075 & 0.240 & 0.420 & 0.595 & 0.760 & 0.930 & 1.100 & 1.270 & 1.445 & 1.620 \\
\cline{2-12}
& $m$ & 11 & 12 & 13 & 14 & 15 & 16 & 17 & 18 & & \\
\cline{2-12}
& $t$ / s & 1.795 & 1.965 & 2.135 & 2.305 & 2.470 & 2.640 & 2.810 & 2.975 & & 
\end{tabular}
\caption{Abgelesene Zeitpunkte für die Extrema der Schwebungsschwingung}
\label{tabTimeSch}
\end{table}
Wir erwarten wiederum einen linearen Zusammenhang $t(m) = \frac T2m + c$ mit der Periodendauer $T$ der Schwebungschwingung. Wir führen also eine lineare Regression durch mit den Resultaten in Tabelle \ref{tabRegressionSch}.

\begin{table}[H]
\centering
\begin{tabular}{c|c|c|c}
Schwebung & $T$ / ms & c & $\chi^2$ / $n_{df}$ \\
\hline
1 & $ 0.6266 \pm 0.0011$ & $ -0.161 \pm 0.003$ & 0.56 \\
2 & $ 1.3060 \pm 0.0063$ & $ -0.28 \pm 0.01$ & 1.43 \\
4 & $ 0.34211 \pm 0.00045$ & $ -0.094 \pm 0.003$ & 1.02
\end{tabular}
\caption{Ergebnisse der Regression für die Schwebungsschwingung}
\label{tabRegressionSch}
\end{table}
Da wir bei Schwebung 3 nur 2 Extrema vernünftig ablesen konnten, wird hier keine Regression gemacht, sondern die Periodendauer mittels $T = 2(t_2-t_1) = 3.04$ bestimmt. Die Unsicherheit beträgt dabei $\sigma_T = 2\sqrt{2} \sigma_t \approx 0.03$. Für die restlichen Schwebungen sind die Residuengraphen in Abbildung \ref{residuenSch}.

\begin{figure}[H]
	\centering
	\includegraphics[width=\linewidth]{plots/residuenSch.pdf}
	\caption{Residuenplots der Regression für die Schwebungsschwingung}
	\label{residuenSch}
\end{figure}

Auf den ersten Blick sehen diese gut aus und es lässt sich keine Systematik erkennen.


Nun wollen wir uns dem Frequenzspektrum aus der FFT widmen. In Abbildung \ref{freqSpecSch} ist beispielhaft das Spektrum einer Schwebung abgebildet. Man sieht deutlich zwei Peaks bei etwa 145 Hz. Diese setzen sich zudem kleiner werdend bei den Vielfachen dieser Frequenz fort.
\begin{figure}[H]
	\centering
	\includegraphics[width=\linewidth]{plots/schwebungsspektrum.pdf}
	\caption{Frequenzspektrum der ersten Messung (\mbox{``schwebung\_1.lab''})}
	\label{freqSpecSch}
\end{figure}

Wir benutzen nun die Peakfinder-Methode der Praktikumsbibliothek um die Frequenzen der Peaks zu bestimmen. Die Resultate sind in Abbildung \ref{fftPeakSch} für alle vier Schwebungen zu sehen.

\begin{figure}[H]
	\centering
	\includegraphics[width=\linewidth]{plots/FFT_freq.pdf}
	\caption{FFT der Schwebungen mit Peakanalyse}
	\label{fftPeakSch}
\end{figure}
Als Fehler wird dabei die Frequenzdifferenz zwischen zwei Punkten in der FFT genommen. Die Ergebnisse der Berechnungen sind in Tabelle \ref{tabFFTpeakSch} aufgelistet. Zudem ist dort auch angegeben auf welche Frequenzbereiche des Spektrums die Peakfinder-Methode angewendet wurde.

\begin{table}[H]
\centering
\begin{tabular}{c|c|c|c|c}
Schwebung & $f_1$ / Hz & Bereich für $f_1$ / Hz & $f_2$ / Hz & Bereich für $f_2$ / Hz \\
\hline
1 & $ 143.51 \pm 0.31$ & 142-145 & $ 146.61 \pm 0.31$ & 145-148 \\
2 & $ 144.92 \pm 0.12$ & 144-145.5 & $ 146.45 \pm 0.12$ & 145.5-147.5 \\
3 & $ 145.85 \pm 0.12$ & 144-146.2 & $ 146.47 \pm 0.12$ & 146.1-148 \\
4 & $ 140.40 \pm 0.12$ & 138-143.5 & $ 146.53 \pm 0.12$ & 143.5-149 
\end{tabular}
\caption{Ergebnisse der Peakanalyse der Frequenspektren}
\label{tabFFTpeakSch}
\end{table}
Mit diesen Werten lassen sich nun die Frequenz der resultierenden Schwingung und der Schwebungsschwingung berechnen, wenn man
$$f_S = \frac{f_2-f_1}2 \, , \hspace{1cm} f_R = \frac{f_2+f_1}2$$
benutzt. Berechnet man zusätzlich anhand der abgelesenen Periodendauern die Frequenzen mittels $f = \frac 1T$, so ergibt sich das Resultat in Tabelle \ref{tabFreqErg}.


\begin{table}[H]
\centering
\begin{tabular}{c|c|c|c|c}
& \multicolumn{2}{c|}{Abgelesen} & \multicolumn{2}{c}{FFT} \\
\hline
Schwebung & $f_S$ / Hz & $f_R$ / Hz & $f_S$ / Hz & $f_R$ / Hz \\
\hline
1 & $ 1.5959 \pm 0.0028$ & $ 144.9290 \pm 0.0074$ & $ 1.55 \pm 0.22$ & $ 145.06 \pm 0.22$ \\
2 & $ 0.7657 \pm 0.0037$ & $ 145.6978 \pm 0.0076$ & $ 0.77 \pm 0.08$ & $ 145.69 \pm 0.08$ \\
3 & $ 0.3289 \pm 0.0032$ & $ 146.164 \pm 0.016$ & $ 0.31 \pm 0.08$ & $ 146.16 \pm 0.08$ \\
4 & $ 2.9230 \pm 0.0038$ & $ 143.426 \pm 0.013$ & $ 3.07 \pm 0.08$ & $ 143.47 \pm 0.08$ 
\end{tabular}
\caption{Ergebnisse der Frequenzen von resultierender und Schwebungs-Schwingung}
\label{tabFreqErg}
\end{table}



\subsubsection{Fazit}

Wir haben nun auf zwei verschiedene Arten Werte für die Frequenzen der Schwebungschwingung und der resultierenden Schwingung erhalten. Zum Vergleich der Werte berechnen wir die relativen Abweichungen.

\begin{table}[H]
\centering
\begin{tabular}{c|c|c|c|c|c|c|c|c}
& \multicolumn{4}{c|}{$f_S$} & \multicolumn{4}{c}{$f_R$} \\
\hline
Schwebung & 1 & 2 & 3 & 4 & 1 & 2 & 3 & 4 \\
\hline
$\frac{|f_1-f_2|}{\sqrt{\sigma_1^2+\sigma_2^2}}$ & 0.21 & 0.05 & 0.24 & 1.84 & 0.60 & 0.10 & 0.05 & 0.54 
\end{tabular}
\end{table}
Die Werte stimmen überwiegend gut überein. Bei Schwebung 4 liegen die Schwebungsfrequenzen $f_S$ weiter auseinander. Es könnte der Fehler bei der Auswertung mit der FFT unterschätzt worden. Besonders die Bestimmung des Peaks ist dabei fraglich. Ein Zusammenhang zur Verwendung von verschiedenen Messungen bei der Auswertung kann ausgeschlossen werden, da eine FFT der anderen Messung diegleichen Frequenzen liefert. 


\subsection{Aufnahme eines Frequenzspektrums}

\subsubsection{Versuchsdurchführung}

In diesem Teilversuch wird das Frequenzspektrum der Gitarre in Abhängigkeit des Anschlagpunktes auf der Saite untersucht. Dazu wird die E-Saite an drei verschiedenen Punkten angeschlagen. Die Saite wird dabei bei $d=\frac 12$, $d=\frac 15$ und $d= \frac 13$ der Saitenlänge angeschlagen. Für den ersten und dritten Anschlagpunkt werden zwei Messungen aufgenommen und für den zweiten eine Messung. Die Messparameter des Sensor-CASSY sind dabei, abgesehen vom Messintervall, wie im ersten Teilversuch. Für die erste Messung liegt das Messintervall bei 200 $\mu$s und für die restlichen bei 100 $\mu$s.


\subsubsection{Versuchsauswertung}

Jeweils eines der Freqeunzspektren für jeden Anschlagpunkt $d$ ist in Abbildung \ref{anschlagSpec} geplottet. Man sieht jeweils gut, dass der Peak für die $1/d$-te Harmonische deutlich weniger ausgeprägt ist, als die Peaks der benachbarten Harmonischen. Bei den Vielfachen ist dies auch noch leicht zu erkennnen. Das die Peaks trotzdem vorhanden sind hängt vermutlich mit einem unsauberen Anschlag zusammen. Theoretisch könnte man die Peakhöhen ablesen und durch Überlagerung der Moden mit den entsprechenden Amplituden die ursprüngliche Auslenkung der Saite rekonstruieren. Das Auftreten eines Peaks bei etwa 3800 Hz in allen Spektren kommt sicher nicht von der Gitarre. Allerdings fehlt uns auch eine andere Erklärung desssen.
\begin{figure}[h]
	\centering
	\includegraphics[width=\linewidth]{plots/anschlagspektrum.pdf}
	\caption{Frequenzspektren bei verschiedenen Anschlagpunkten}
	\label{anschlagSpec}
\end{figure}





